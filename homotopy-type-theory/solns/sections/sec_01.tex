\setcounter{subsection}{1-1}
\subsection{Exercises}
% \exercise{1}{
% Given functions $f : A \to B$ and $g : B \to C$, define their
% \textbf{composite} $g \circ f : A \to C$. Show that we have $h \circ (g \circ f ) \equiv (h \circ g) \circ f$.}
%
% \exercise{2}{
%   Derive the recursion principle for products  $\text{rec}_{A \times B}$  using
% only the projections, and verify that the definitional equalities are valid.
% Do the same for $\Sigma$-types.
% }
%
% \exercise{3}{
%   Derive the induction principle for products $\text{ind}_{A \times B}$, using
% only the projections and the propositional uniqueness principle uppt.
% Verify that the definitional equalities are valid. Generalize uppt to $\Sigma$-
% types, and do the same for $\Sigma$-types. (This requires concepts from Chapter 2.)
% }
%
% \exercise{4}{
% Assuming as given only the iterator for natural numbers
% }
%
% \exercise{4}{
%
% }


\exercise{1}
  Given functions $f:A\to B$ and $g:B\to C$, define their \define{composite}
  \indexdef{composition!of functions}%
  \indexdef{function!composition}%
 $g\circ f:A\to C$.
  \index{associativity!of function composition}%
  Show that we have $h \circ (g\circ f) \jdeq (h\circ g)\circ f$.


\exercise{1}
  Derive the recursion principle for products $\rec{A\times B} $ using only the projections, and verify that the definitional equalities are valid.
  Do the same for $\Sigma$-types.


\exercise{1}
  Derive the induction principle for products $\ind{A\times B}$, using only the projections and the propositional uniqueness principle $\uniq{A\times B}$.
  Verify that the definitional equalities are valid.
  Generalize $\uniq{A\times B}$ to $\Sigma$-types, and do the same for $\Sigma$-types.
  % \emph{(This requires concepts from \cref{cha:basics}.)}


\exercise{1}
\index{iterator!for natural numbers}
Assuming as given only the \emph{iterator} for natural numbers
\[\ite : \prd{C:\UU} C \to (C \to C) \to \nat \to C \]
with the defining equations
\begin{align*}
\ite(C,c_0,c_s,0)  &\defeq c_0, \\
\ite(C,c_0,c_s,\suc(n)) &\defeq c_s(\ite(C,c_0,c_s,n)),
\end{align*}
derive a function having the type of the recursor $\rec{\nat}$.
Show that the defining equations of the recursor hold propositionally for this function, using the induction principle for $\nat$.


\exercise{1}
\index{type!coproduct}%
Show that if we define $A + B \defeq \sm{x:\bool} \rec{\bool}(\UU,A,B,x)$, then we can give a definition of $\ind{A+B}$ for which the definitional equalities stated in \cref{sec:coproduct-types} hold.


\exercise{1}
\index{type!product}%
Show that if we define $A \times B \defeq \prd{x:\bool}\rec{\bool}(\UU,A,B,x)$, then we can give a definition of  $\ind{A\times B}$ for which the definitional equalities stated in \cref{sec:finite-product-types} hold propositionally (i.e.\ using equality types).
\emph{(This requires the function extensionality axiom, which is introduced in \cref{sec:compute-pi}.)}


\exercise{1}
Give an alternative derivation of $\indidb{A}$ from $\indid{A}$ which avoids the use of universes.
  \emph{(This is easiest using concepts from later chapters.)}


\exercise{1}
  \index{multiplication!of natural numbers}%
  Define multiplication and exponentiation using $\rec{\nat}$.
  Verify that $(\nat,+,0,\times,1)$ is a semiring\index{semiring} using only $\ind{\nat}$.
  You will probably also need to use symmetry and transitivity of equality, \cref{lem:opp,lem:concat}.


\exercise{1}
  \index{finite!sets, family of}%
  Define the type family $\Fin : \nat \to \UU$ mentioned at the end of \cref{sec:universes}, and the dependent function $\fmax : \prd{n:\nat} \Fin(n+1)$ mentioned in \cref{sec:pi-types}.


\exercise{1}
  \indexdef{function!Ackermann}%
  \indexdef{Ackermann function}%
  Show that the Ackermann function $\ack : \nat \to \nat \to \nat$ is definable using only $\rec{\nat}$ satisfying the following equations:
  \begin{align*}
    \ack(0,n) &\jdeq \suc(n), \\
    \ack(\suc(m),0) &\jdeq \ack(m,1), \\
    \ack(\suc(m),\suc(n)) &\jdeq \ack(m,\ack(\suc(m),n)).
  \end{align*}


\exercise{1}
  Show that for any type $A$, we have $\neg\neg\neg A \to \neg A$.


\exercise{1}
  Using the propositions as types interpretation, derive the following tautologies.
  \begin{enumerate}
  \item If $A$, then (if $B$ then $A$).
  \item If $A$, then not (not $A$).
  \item If (not $A$ or not $B$), then not ($A$ and $B$).
  \end{enumerate}


\exercise{1}
  Using propositions-as-types, derive the double negation of the principle of excluded middle, i.e.\ prove \emph{not (not ($P$ or not $P$))}.


\exercise{1}
  Why do the induction principles for identity types not allow us to construct a function $f: \prd{x:A}{p:\id{x}{x}} (\id{p}{\refl{x}})$ with the defining equation
  \[ f(x,\refl{x}) \defeq \refl{\refl{x}} \quad ?\]


\exercise{1}
  Show that indiscernibility of identicals follows from path induction.  


\exercise{1}
  Show that addition of natural numbers is commutative: $\prd{i,j:\nat} (i+j=j+i)$.

